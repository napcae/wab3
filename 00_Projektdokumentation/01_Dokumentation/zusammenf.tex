%% zusammenf.tex
%% $Id: zusammenf.tex 61 2012-05-03 13:58:03Z bless $
%%
% !TEX root = thesis.tex

\chapter{Zusammenfassung und Ausblick}
\label{ch:Zusammenfassung}
%% ==============================

Diese Analyse liefert auf Basis einer ausführlichen Grundlage an Daten mit den gängigen Verfahren der Datenanalyse und -auswertung, der zuvor gewonnenen Daten, ein sehr exaktes Bild über die Zusammenhänge der jeweiligen Aktivitäten.
Die graphische Darstellung orientiert sich strikt an der Leitfrage des Projekts. 
Dadurch erhält der Leser eine erste Orientierung in welche Richtung die Beantwortung der Leitfrage ziehlen wird.
\\
Ob sich das Wohlbefinden durch Quantified Self verbessern lässt, kann man nicht durch Zahlen belegen. 
Es hängt von einem Selbst ab, inwiefern man sich damit besser fühlt und von den Daten beeinflussen lässt.
Ein weiterer Punkt ist die Anregung bzw. Motivation, die man durch interne Konkurrenz während des Projekts, beispielsweise durch höhere „Scores“ für Schritt und Kalorienzähler zu erreichen versucht. \\
Ziel von QS ist es nicht das Leben direkt zu verbessern, sondern dass es dem Nutzer besser geht. \\
Insofern lässt sich die Frage, ob sich das Wohlbefinden durch QS verbessern lässt, schlussendlich mit einem „Ja“ beantworten.
Dabei ist es allerdings wichtig nicht die Motivation zu verlieren und Auswertungen wie z.B. den Abbildungen [\ref{fig:SCEffect}] und [\ref{fig:SCQualityPerDay}] einzugehen.\\
Mit diesen Informationen lässt sich feststellen, welche Faktoren Einfluss auf das persönliche Leben haben, um an diesen im Anschluss gezielt zu arbeiten.
Dadurch kann QS das Leben im Alltag positiv beeinflussen.
Ob QS wirklich dazu beiträgt, dass man Gesünder wird und das Gesundheitssystem entlasten kann, ist eine andere frage – doch gefühlt steigert sich das Wohlbefinden des Nutzers.

%%% Local Variables: 
%%% mode: latex
%%% TeX-master: "thesis"
%%% End: 
