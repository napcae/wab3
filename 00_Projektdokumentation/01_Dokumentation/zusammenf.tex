%% zusammenf.tex
%% $Id: zusammenf.tex 61 2012-05-03 13:58:03Z bless $
%%

\chapter{Zusammenfassung und Ausblick}
\label{ch:Zusammenfassung}
%% ==============================

Diese Analyse liefert auf Basis einer ausführlichen Datenanalyse mit den gängigen Verfahren der Datenanalyse und -auswertung, der zuvor gewonnenen Daten, ein sehr exaktes Bild über die Zusammenhänge der jeweiligen Aktivitäten.
Die graphische Darstellung orientiert sich strikt an der Leitfrage des Projekts. 
Dadurch erhält der Leser eine erste Orientierung in welche Richtung die Beantwortung der Leitfrage ziehlen wird.

Fazit:
Ob sich das Wohlbefinden durch Quantified Self verbessern lässt, kann man nicht durch Zahlen belegen. Es hängt von einem Selbst ab, inwiefern man sich damit besser fühlt.
*anregung motivation, interne konkurrenz(mehr laufen im projekt), ob es wirklich dazu beiträgt dass man gesünder ist ist eine andere frage, aber gefühlt fühlt man sich besser.
ziel von QS ist es nicht das leben zu verbessern, sonder dass man sich selbst besser fühlt. in sofern lässt sich die frage ob sich das wohlbefinden durch QS verbesern lässt mich ja beantworten.

%%% Local Variables: 
%%% mode: latex
%%% TeX-master: "thesis"
%%% End: 
