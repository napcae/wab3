%% grundlagen.tex
%% $Id: grundlagen.tex 61 2012-05-03 13:58:03Z bless $
%%

\chapter{Grundlagen}
\label{ch:Grundlagen}
%% ==============================
Die Grundlagen müssen soweit beschrieben werden, dass ein Leser das Problem und die Problemlösung versteht.
Um nicht zu viel zu beschreiben, kann man das auch erst gegen Ende der Arbeit schreiben.


%% ==============================
\section{Quantified Self}
%% ==============================
\label{ch:Grundlagen:sec:QuantifiedSelf}

\subsection{Allgemein}
%% ==============================
\label{ch:Grundlagen:sec:QuantifiedSelf:subsec:Allgemein}

Quantified Self ist ein Ausdruck dafür, Technologien oder Anwendungen zu nutzen, um das tägliche Leben durch ständige Datenerfassung zu visualisieren. 
Hauptsächliche Bereiche, die durch Quantified Self erfasst werden sollen sind z.B. Essgewohnheit (welche Lebensmittel verzehrt wurden), persönliches Wohlbefinden (Stimmung, Erregung, Sauerstoffgehalt im Blut) und die Leistung (mentale und physische). \cite{web:WhatIsQS}\\
Umgangssprachlich wird eine solche Selbstüberwachung und -erkennung, die tragbare Sensoren (EEG, ECG-, Video-, etc.)\textbf{(!erklären)} und mobile Plattformen (tragbare Fitness-Gadgets, Smartphones/Tablets, etc.) verbindet, auch „Self-Tracking“ genannt. 
Durch den heutigen Stand der Technik ist es so jedem möglich, bisher unbekannte eigene biometrische Daten kostengünstig und bequem zu ermitteln.

\subsection{Einsatzgebiete}
%% ==============================
\label{ch:Grundlagen:sec:QuantifiedSelf:subsec:Einsatzgebiete}

Der Hauptanwendungsbereich von Quantified Self ist die Verbesserung der eigenen Gesundheit und des persönlichen Wohlbefindens. 
Für diesen Bereich gibt es viele Geräte und Applikationen, die die körperliche Aktivität, die Kalorienzufuhr, die Schlafqualität, die Körperhaltung und andere Faktoren des persönlichen Wohlbefindens analysieren und helfen, die gewonnenen Daten visuell verständlich darzulegen. 
Diese Gesundheitsüberwachung soll das persönliche Wohlbefinden des Nutzers aufrecht erhalten und so potentielle Krankheiten verhindern.
Die daraus resultierenden sinkenden Gesundheitskosten sind vor allem für Nutzer in Ländern ohne öffentliches Gesundheitssystem eine willkommene Entwicklung.
\\
\\
Ein weiteres Anwendungsgebiet findet sich in der Bildung. 
So nutzen viele Schulen, vor allem in den USA(United States of America), QS-Applikationen, um den Schülern „schwierige“ Fächer praxisnah beizubringen. 
So werden dort die Aktivitäten der Schüler aufgenommen und ausgewertet.
Aus den gewonnenen Daten werden themenrelevante Gebiete aus der Mathematik und den Naturwissenschaften den Schülern direkt auf das Smartphone geliefert.\cite{web:QSEducation}

\ldots 


\subsection{Ausblick}
%% ==============================
\label{ch:Grundlagen:sec:QuantifiedSelf:subsec:Ausblick}

Quantified Self steckt in Deutschland noch in den Kinderschuhen.
Die Bewegung des „Self-Tracking” wächst langsam aber stetig weiter an und entwickelt sich weiter.
So ist es durchaus denkbar, dass aus Quantified Self ein „Quantified Us“ in naher Zukunft entstehen wird.
Da das eigene Tracking eine starke Disziplin des Nutzers verlangt und keinen Vergleich mit anderen Personen zulässt, wäre die Entwicklung zu einem „Us“ nur zu gut nachvollziehbar.
\\
Man stelle sich beispielsweise eine Person vor, die an Epilepsie erkrankt ist. 
Diese versucht die Steigerung seiner Anfälle mit Hilfe des Trackings (beispielsweise Bewegungsmessungen) zu verstehen und einen eventuellen Zusammenhang mit der Umwelt zu finden. \\	
Könnte diese Person die ausgewerteten Daten besser verstehen, wenn sie im Vergleich zu den Daten anderen Personen mit der selben Krankheit stünden? \\
Eine solche Möglichkeit könnte den Nutzern sehr wohl weiterbringen, da dies erstens andere Geräte und Anwendungen überflüssig macht und zweitens dem Nutzer verdeutlicht, wie seine Anfallhäufigkeit im Durchschnitt zu den der anderen Patienten liegt.
Das gepaart mit der Kontaktaufnahme zu anderen Betroffenen könnte einen zusätzlichen Schub an Motivation mit sich bringen. 
So wird die letztendliche Form des „Quantified Self” das „Quantified Us” werden, ein soziales “Self-Tracking“ Netzwerk, das den Austausch zwischen erkrankten Personen maßgeblich erleichtern und ein Gemeinschaftsgefühl unter diesen herstellen kann.


% %% ==============================
% \section{Abschnitt 2}
% %% ==============================
% \label{ch:Grundlagen:sec:Abschnitt2}

% Bla fasel\ldots

%% ==============================
\section{Verwandte Arbeiten}
%% ==============================
\label{ch:Grundlagen:sec:RelatedWork}
Hier kommt "`Related Work"' rein.
Eine Literaturrecherche sollte so vollständig wie möglich sein, relevante Ansätze müssen beschrieben werden und es sollte deutlich gemacht werden, wo diese Ansätze Defizite aufweisen oder nicht anwendbar sind, z.\,B. weil sie von anderen Umgebungen oder Voraussetzungen ausgehen.


Bla fasel\ldots

%%% Local Variables: 
%%% mode: latex
%%% TeX-master: "thesis"
%%% End: 
