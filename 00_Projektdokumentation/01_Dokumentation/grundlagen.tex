%% grundlagen.tex
%% $Id: grundlagen.tex 61 2012-05-03 13:58:03Z bless $
%%

\chapter{Grundlagen}
\label{ch:Grundlagen}
%% ==============================
Die Grundlagen müssen soweit beschrieben
werden, dass ein Leser das Problem und
die Problemlösung  versteht.Um nicht zuviel 
zu beschreiben, kann man das auch erst gegen 
Ende der Arbeit schreiben.

Bla fasel\ldots

%% ==============================
\section{Quantified Self}
%% ==============================
\label{ch:Grundlagen:sec:Quantified Self}

\subsection{Allgemein}
%% ==============================
\label{ch:Grundlagen:sec:Quantified Self:subsec:Allg}

Quantified Self ist ein Ausdruck dafür, Technologien oder Anwendungen zu nutzen, um das tägliche Leben durch ständige Datenerfassung zu visualisieren.
Hauptsächliche Bereiche, die durch Quantified Self erfasst werden sollen sind z.B. Essgewohnheit (welche Lebensmittel verzehrt wurden), persönliches Wohlbefinden (Stimmung, Erregung, Sauerstoffgehalt im Blut) und die Leistung (mentale und physische). 
Umgangssprachlich wird eine solche Selbstüberwachung und -erkennung, die tragbare Sensoren (EEG, ECG-, Video-, etc.) und mobile Plattformen (tragbare Fittnes-Gadgets, Smartphones/Tablets, etc.) verbindet, auch "self-tracking" genannt. 
Durch den heutigen Stand der Technik ist es so jedem möglich, bisher unbekannte eigene biometrische Daten kostengünstig und bequem zu ermitteln.

\subsection{Einsatzgebiete}
%% ==============================
\label{ch:Grundlagen:sec:Quantified Self:subsec:Einsatz}

Bla fasel\ldots

\subsection{Ausblick}
%% ==============================
\label{ch:Grundlagen:sec:Quantified Self:subsec:Ausblick}

Bla fasel\ldots



%% ==============================
\section{Abschnitt 2}
%% ==============================
\label{ch:Grundlagen:sec:Abschnitt2}

Bla fasel\ldots

%% ==============================
\section{Verwandte Arbeiten}
%% ==============================
\label{ch:Grundlagen:sec:RelatedWork}
Hier kommt "`Related Work"' rein.
Eine Literaturrecherche sollte so vollständig wie möglich sein,
relevante Ansätze müssen beschrieben werden und es sollte deutlich 
gemacht werden, wo diese Ansätze Defizite aufweisen oder nicht
anwendbar sind, z.\,B. weil sie von anderen Umgebungen oder 
Voraussetzungen ausgehen.


Bla fasel\ldots

%%% Local Variables: 
%%% mode: latex
%%% TeX-master: "thesis"
%%% End: 
