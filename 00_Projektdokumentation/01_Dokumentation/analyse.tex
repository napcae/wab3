%% analyse.tex
%% $Id: analyse.tex 61 2012-05-03 13:58:03Z bless $

\chapter{Analyse und Evaluierung}
\label{ch:AnalyseUndEvaluierung}
%% ==============================

Die Beantwortung der Leitfrage des Projekts, ob durch Quantified Self das Wohlbefinden verbessert werden kann, hängt maßgeblich von einer genauen Analyse der Daten ab. 
Die zuvor in der Datengenerierungsphase erhobenen Daten müssen unter Berücksichtigung etwaiger Fremdeinflüsse bzw. Verfälschungen analysiert und ausgewertet werden.
Um dies zu gewährleisten sind wichtige Anforderungen an die Analyse gestellt, auf die im Folgenden näher eingegangen wird.

Exemplarische wurden nun nachdem alle Daten analysiert und ausgewertet wurden eine Testperson ausgewählt, anhand der die Leitfrage „Lässt sich das Wohlbefinden durch Quantified Self verbessern?“ beantwortet werden soll.

Aufgrund einer in der Gesamtheit Neuerhebung von Daten, liegt bei deren Auswertung der Fokus auf der deskriptive Datenanalyse.

\begin{quote}
Deskriptive Datenanalyse: Liegt eine Totalerhebung oder generell ein Datensatz vor, so ist es die Aufgabe der Datenanalyse, die in den Einzeldaten enthaltene Information zu verdichten und diese so darzustellen, dass Wesentliches deutlich wird. Dazu werden Tabellen, graphische Darstellungen und charakteristische Maßzahlen verwendet.  Die Datenanalyse hat ausschließlich beschreibenden Charakter (deskriptive Statistik). 
\end{quote}
\cite{http://wirtschaftslexikon.gabler.de/Definition/datenanalyse.html?referenceKeywordName=statistische+Datenanalyse}


Laut Schäfer\cite{Schafer2010} ist im Anschluss der deskriptiven Datenanalye mit der explorativen Statistik fortzufahren.
Dabei wird versucht Muster zu erkennen, welche mit Hilfe von Grafiken und Daten beschrieben werden.
Abschließend wird mit der Inferenzstatistik die Auswertung vollendet.
In diesem letzten Schritt wird versucht mit Hilfe von Stichprobendaten auf die allgemeine These zu schließen.

Dazu wurden verschiedene Korrelationskoeffizienten berechnet und Graphen erstellt.




%% ==============================
\section{Zusammenhang von Schlafqualität und Schritten am Tag}
%% ==============================
\label{ch:AnalyseUndEvaluierung:sec:ZusammenhangVonSchlafqualitätUndSchrittenAmTag}

\begin{figure}[H]
\centering
        \includegraphics[width=0.5\textwidth]{images/Analyse} 
        \caption[xxx]{xxx}
        \label{fig:xxxx}
\end{figure}

%% ==============================
\section{Zusammenhang von Stimmung und Schlafqualität}
%% ==============================
\label{ch:AnalyseUndEvaluierung:sec:KorrelationVonSchlafqualitätUndSchrittenAmTag}

\begin{figure}[H]
\centering
        \includegraphics[width=0.2\textwidth]{images/Analyse} 
        \caption[xxx]{xxx}
        \label{fig:xxxx}
\end{figure}

%% ==============================
\section{Zusammenhang von Schlafqualität und Schritten am Tag}
%% ==============================
\label{ch:AnalyseUndEvaluierung:sec:KorrelationVonSchlafqualitätUndSchrittenAmTag}

\begin{figure}[H]
\centering
        \includegraphics[width=0.2\textwidth]{images/Analyse} 
        \caption[xxx]{xxx}
        \label{fig:xxxx}
\end{figure}

%% ==============================
\section{Zusammenfassung}
%% ==============================
\label{ch:Analyse:sec:zusammenfassung}

Diese Analyse liefert auf Basis einer ausführlichen Datenanalyse mit den gängigen Verfahren der Datenanalyse und -auswertung, der zuvor gewonnenen Daten, ein sehr exaktes Bild über die Zusammenhänge der jeweiligen Aktivitäten.
Die graphische Darstellung orientiert sich strikt an der Leitfrage des Projekts. 
Dadurch erhält der Leser eine erste Orientierung in welche Richtung die Beantwortung der Leitfrage ziehlen wird.

%%% Local Variables: 
%%% mode: latex
%%% TeX-master: "thesis"
%%% End: 
