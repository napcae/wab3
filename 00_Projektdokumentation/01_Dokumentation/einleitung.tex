%% Einleitung.tex
%% $Id: einleitung.tex 61 2012-05-03 13:58:03Z bless $
%% 
% !TEX root = thesis.tex

\chapter{Einleitung}
\label{ch:Einleitung}
%% ==============================

%% ==============================
\section{Problemstellung}
%% ==============================
\label{ch:Einleitung:sec:Problemstellung}

Seit Gründung der Initiative Quantified Self(QS)\cite{web:QS} im Jahr 2007\cite{web:QSJahr} steigen die Möglichkeiten von Jahr zu Jahr, um Umwelt und personenbezogene Daten zu erfassen\cite{web:Tracking}. 
Dies wird durch unterschiedliche Hard- und Softwarelösungen ermöglicht. \\
Dabei werden Erkenntnisse über Gesundheit, Fitness und persönliche Stimmung gesammelt.
Diese können auch mit externen Umweltfaktoren in Relation gebracht werden. \\
Als Leitthema des Projekts wurde die Frage, ob durch Quantified Self das Wohlbefinden verbessert werden kann, festgesetzt. 
Das Erreichen des Ziels, die Beantwortung der Leitfrage, soll durch das Analysieren und Auswerten von aus Selbstversuchen gewonnenen Daten geschehen.
Zur Zielerreichung wird zu Beginn der Datengenerierungs- bzw. Testphase, die 30 Tage beträgt, der augenblickliche Zustand der Probanden mit der Hilfe der wie in Kapitel \ref{ch:Apps} beschrieben Applikationen aufgenommen. 
% Zustand der Probanden aufgezeichnet und gesichert –– also der derzeitige Schlafrhythmus, die derzeitige Essgewohnheit und die Bewegungsaktivität.
% Dieser wird als 100\% Marke angesetzt und dient der späteren Auswertung der gewonnen Daten als Maßstab. \\
Die Daten werden aus Bewegungsaktivität, Schlafrhythmus und Stimmungslage gewonnen.
% Sollte der analysierte Wert nach der Testphase über dieser Marke liegen, liegt eindeutig eine Verbesserung vor. 
Diese Richtwerte dienen der Orientierung, um die Leitfrage anhand der über den gesamten Projektzeitraum gewonnen Daten zu beantworten. \\
Liegen die Werte deutlich unter den Ausgangszustand, wird von einer Verschlechterung des Wohlbefinden ausgegangen. \\
% Zur besseren Klassifizierung der Daten wird von einer Verbesserung erst ab dem Wert von mindestens 120\% gesprochen, sowie von einer Verschlechterung bei einem Wert von 80\%. \\Sollte der Endwert eines Probanden zwischen 80\% und 120\% liegen wird von einem Gleich bleiben des Befinden gesprochen.\\ 
In der heutigen Zeit, in der die Lebenssituation, vor allem in der arbeitenden Bevölkerung, an Qualität abnimmt – sei es durch Stress im Arbeitsalltag oder der gewaltigen Informationsflut, die uns unterbewusst immer und überall beeinträchtigt – ist es wichtig, neue Möglichkeiten auszuloten, um die Lebensqualität zum Beispiel durch die Selbstanalyse diverser Faktoren wieder zu verbessern.  

\section{Zielsetzung}
%% ==============================
\label{ch:Einleitung:sec:Zielsetzung}

In diesem Projekt werden Faktoren wie Schlaf- und Bewegungsaktivität sowie persönliche Stimmung täglich ermittelt, die mit Hilfe von Quantified Self Applikationen für das Smartphone aufgezeichnet und später analysiert werden. \\
Dadurch soll herausgefunden werden, ob eine Verbesserung des Wohlbefinden durch die Nutzung von QS-Applikationen möglich ist. \\
Die stetig steigende Anzahl von Burnout-Patienten\cite[Oberlander]{Oberlander:Burnout} und die Selbsteinschätzung vieler Menschen in Deutschland, die entgegen dem eigentlichen Trend, eine sinkende Lebensqualität bemängeln, versuchen wir mit unserem Projekt eine Perspektive zu geben, wie eventuell die Situation durch den Einsatz mobiler QS-Applikation für diverse Faktoren verbessern kann. \\
Dieses Projekt soll eventuelle neue Möglichkeiten zur Verbesserung des Wohlbefinden durch das Nutzen von QS aufzeigen und helfen den Burnout zu verhindern beziehungsweise(bzw.) Stress abzubauen und so das Gesundheitssystem teilweise entlasten, sowie das Lebensgefühl verbessern. 

%% ==============================
\section{Gliederung der Arbeit}
%% ==============================
\label{ch:Einleitung:sec:GliederungDerArbeit}

Die Arbeit ist in folgende sieben Teile gegliedert:

\begin{enumerate}
\def\labelenumi{\arabic{enumi}.}
\itemsep1pt\parskip0pt\parsep0pt
\item
  Einleitung
\item
  Grundlagen zu Quantified Self
\item
  Softwarebeschreibung

  \begin{enumerate}
  \def\labelenumii{\alph{enumii}.}
  \itemsep1pt\parskip0pt\parsep0pt
  \item
    Moves
  \item
    Hueman
  \item
    Sleep Cycle
  \end{enumerate}
\item
  Relativierung: Mögliche Fehlerquellen
\item
  Analyse der ausgewerteten Daten
\item
  Fazit
\end{enumerate}

Die Einleitung soll einen Einblick in die Problemstellung und Zielsetzung der Arbeit, Motivation und Trend, sowie den Aufbau der Arbeit beschreiben.
Informationen zu Quantified Self gibt Aufschluss über aktuelle Studien zu Quantified Self sowie den Trend und Medizinische Untersuchungen.
Innerhalb die Softwarebeschreibung wird detaillierter auf die Auswahl der Apps eingegangen.\\
Zusätzlich sind deren Funktionsweise und Features hier beschrieben.
Die Relativierung beschreibt mögliche technische und persönliche Fehlerquellen bei der Anwendung, sowie die Problematik der falschen Wahrnehmung der eigenen Verfassung.
\\
Im Hauptteil, der Auswertung und Analyse der gewonnen Daten soll mit Hilfe statistischer Methoden die These überprüft werden.
\\
Das Fazit beantwortet die Leitfrage des Projektes und soll Aufschluss über mögliche Ideen der Verbesserung geben.

%% ==============================
\section{Auswahl der Trackingmethoden}
%% ==============================
\label{ch:Einleitung:sec:AuswahlDerTrackingmethoden}

Aufgrund des geringen finanziellen Budgets und dem Ziel die Fragestellung möglichst realitätsnah zu beantworten, werden die QS Tracking-Methoden auf reine Softwarelösungen beschränkt. 
Diese können mit etlichen handelsüblichen Smartphones benutzt werden und liefern für geringen Kapitals Aufwand gute Ergebnisse\cite{web:TrackingResults,web:AppPreis}. \\
Die Arbeit soll sich an üblichen Situtationen des Alltags orientieren. 
Daher benutzt das Projektteam den Schrittzähler die Software „Moves”, zur Schlafzykluserfassung „Sleep Cycle” und als Stimmungsbarometer „Hueman”.
Die Software wird im folgenden Kapitel näher erläutert. \\
Die Zielgruppe, für die dieses Projekt ins Leben gerufen wurde, sind vor allem Smartphone-Nutzer, deren derzeitiges Leben, sei es durch Stress im Arbeitsalltag oder Burnout-ähnlichen Symptomen, verbesserungswürdig ist beziehungsweise die die derzeitige Lebenssituation zu verbessern suchen oder es auch einfach nur analysieren möchten. 

%%% Local Variables: 
%%% mode: latex
%%% TeX-master: "thesis"
%%% End: 
