%% Einleitung.tex
%% $Id: einleitung.tex 61 2012-05-03 13:58:03Z bless $
%%

\chapter{Einleitung}
\label{ch:Einleitung}
%% ==============================

%% ==============================
\section{Problemstellung}
%% ==============================
\label{ch:Einleitung:sec:problemstellung}

Seit Gründung der Initiative \href{http://quantifiedself.com/}{\textbf{Quantified Self}}(QS) im \href{http://quantifiedself.com/2011/03/what-is-the-quantified-self/}{\textbf{Jahr 2007}} steigen die Möglichkeiten von Jahr zu Jahr, Umwelt und personenbezogene Daten zu erfassen. 
Dies wird durch unterschiedliche Hard und Softwarelösungen ermöglicht. 
Dabei werden Erkenntnisse über Gesundheit, Fitness und persönliche Stimmung gesammelt.
Diese können auch zu externen Umweltfaktoren in Relation gebracht werden.
Als Leitfrage des Projekts wurde die Frage, ob durch Qantified Self das Leben verbessert werden kann, festgesetzt. 
Das Erreichen des Ziels, die Beantwortung der Leitfrage durch das Analysieren und Auswerten von aus Selbstversuchen gewonnener Daten[...].
Zur Zielerreichung wird zu Beginn der Datengenerierungs- bzw. Testphase, die 30 Tage beträgt, der augenblickliche Zustand der Probanden aufgezeichnet und gesichert - also der derzeitige Schlafrhythmus, derzeitige Essgewohnheit und Bewegungsaktivität. %formulierung
Dieser wird als 100\% Marke angesetzt und dient der späteren Auswertung der gewonnen Daten als Maßstab. 
Die Daten werden aus Bewegungsaktivität, Schlafrhythmus und Stimmungslage gewonnen 
Sollte der analysierte Wert nach der Testphase über dieser Marke liegen, liegt eindeutig eine Verbesserung vor. 
Ist der Wert darunter, so stellt dieser eine Verschlechterung dar. 
Zur besseren Klassifizierung der Daten wird von einer Verbesserung erst ab dem Wert von mindestens 120\% gesprochen, sowie von einer Verschlechterung bei einem Wert von 80\%. Sollte der Endwert eines Probanden zwischen 80\% und 120\% liegen wird von einem Gleichbleiben des Befindens gesprochen.
In der heutigen Zeit, in der die Lebenssituation, vor allem in der arbeitenden Bevölkerung, an Qualität abnimmt – sei es durch Stress im Arbeitsalltag oder der gewaltigen Informationsflut, die uns unterbewusst immer und überall beeinträchtigt – ist es wichtig, neue Möglichkeiten auszuloten, um die Lebensqualität zum Beispiel durch die Selbstanalyse diverser Faktoren wieder zu verbessern. 

\section{Zielsetzung}
%% ==============================
\label{ch:Einleitung:sec:Zielsetzung}

In diesem Projekt werden Faktoren wie Schlaf, Ernährung und Bewegungsaktivität sein, die mit Hilfe von Quantified Self Appliaktionen für das Smartphone aufgezeichnet und später analysiert werden. %formulierung
Dadurch soll herausgefunden werden, ob eine Verbesserung durch die Nutzung von QS-Applikationen möglich ist.
Die stetig steigende Anzahl von Burnout-Patienten und die Selbsteinschätzung vieler Menschen in Deutschland, die entgegen dem eigentlichen Trend, eine sinkende Lebensqualität bemängeln, versuchen wir mit unserem Projekt eine Perspektive zu geben, wie man eventuell die Situation durch den Einsatz mobiler QS-Applikation für diverse Faktoren verbessern kann. 
Dieses Projekt soll eventuelle neue Möglichkeiten zur Verbesserung des Lebens durch das Nutzen von QS aufzeigen und helfen den Burnout zu verhindern bzw. Stress abzubauen und so das Gesundheitssystem teilweise entlasten, sowie das Lebensgefühl verbessern. 

%% ==============================
\section{Gliederung der Arbeit}
%% ==============================
\label{ch:Einleitung:sec:gliederung-der-arbeit}

Die Arbeit ist in sieben Teile gegliedert:

\begin{enumerate}
\def\labelenumi{\arabic{enumi}.}
\itemsep1pt\parskip0pt\parsep0pt
\item
  Einleitung (Motivation, Trend)
\item
  Informationen zu Quantified Self (Studien, Trend, Medizinische
  Untersuchung)
\item
  Softwarebeschreibung (Erläuterung, Einführung)

  \begin{enumerate}
  \def\labelenumii{\alph{enumii}.}
  \itemsep1pt\parskip0pt\parsep0pt
  \item
    Moves (Bewegungsaktivität)\\
  \item
    Hueman (pers. Wohlbefinden)\\
  \item
    SleepCycle (Schlafzyklen-Analyse)
  \end{enumerate}
\item
  Relativierung: Mögliche Fehlerquellen (technische, persönliche,
  falsche Wahrnehmung der eigenen Verfassung)
\item
  Auswertung der generierten App-Daten
\item
  Analyse der ausgewerteten Daten
\item
  Fazit (Beantwortung der Leitfrage)
\end{enumerate}

Die Einleitung soll einen Einblick in die Problemstellung und Zielsetzung der Arbeit, Motivation und Trend, sowie den Aufbau der Arbeit beschreiben.
Informationen zu Quantified Self gibt Aufschluss über aktuelle Studien zu Quantified Self sowie den Trend und Medizinische Untersuchungen.
Innerhalb die Softwarebeschreibung wird detailierter auf die Auswahl der Apps eingegangen. 
Zusätzlich sind deren Funktionsweise und Features hier beschrieben.
Die Relativierung beschreibt mögliche technische und persönliche Fehlerquellen bei der Andwendung, sowie die Problematik bei falscher Wahrnehmung der eigenen Verfassung.
Auswertung der generierten App-Daten
Analyse der ausgewerteten Daten
Das Fazit beantwortet die Leitfrage des Projektes und soll Aufschluss über mögliche Verbesserungsideen geben.

%% ==============================
\section{Auswahl der Trackingmethoden}
%% ==============================
\label{ch:Einleitung:sec:auswahl-der-trackingmethoden}

Aufgrund der gegeben Mittel und dem Ziel die Fragestellung realitätsnah zu beantworten, beschränken wir unsere Trackingmethoden auf reine Softwarelösungen. 
Diese können mit etlichen handelsüblichen Smartphones benutzt werden und liefern für weniger als 2\euro{} gute Ergebnisse.(**Belegen**) 
Die Arbeit orientiert sich an alltagsüblichen Situtation. 
Daher benutzt das Projektteam einen Schrittzähler („Moves”), Schlafzykluserfassung („Sleep Cycle”) und einen Stimmungsbarometer („Human”).
Die Software wird im folgenden näher erläutert.
Die Zielgruppe, für die dieses Projekt ins Leben gerufen wurde, sind vor allem Smartphone-Nutzer, deren derzeitiges Leben, sei es durch Stress im Arbeitsalltag oder Burnout-ähnlichen Symptomen, verbesserungswürdig ist bzw. die die derzeitige Lebenssituation zu verbessern suchen(oder es auch einfach nur analysieren möchten).  

%%% Local Variables: 
%%% mode: latex
%%% TeX-master: "thesis"
%%% End: 
