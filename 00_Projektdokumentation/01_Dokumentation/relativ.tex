%% relativ.tex
%% $Id: analyse.tex 61 2012-05-03 13:58:03Z bless $

\chapter{Relativierung}
\label{ch:Relativierung}
%% ==============================

Diese Kapitel beschäftigt sich mit möglichen Fehlerquellen die während der Datenerhebung und Auswertung auftreten können. 
Dabei wird Unterschieden in technische, sowie statistische Schwierigkeiten die während der Aufzeichnungsphase im Projekt auftreten können.




Die Relativierung 

In diesem Kapitel sollten zunächst das zu lösende Problem
sowie die Anforderungen und die Randbedingungen 
einer Lösung\index{Lösung} beschrieben werden (also nochmal
eine präzisierte Aufgabenstellung\index{Aufgabenstellung}).

Dann folgt üblicherweise ein Überblick über bereits existierende
Lösungen bzw. Ansätze, die meistens andere Voraussetzungen bzw.
Randbedingungen annehmen.

Bla fasel\ldots

%% ==============================
\section{Anforderungen}
%% ==============================
\label{ch:Relativierung:sec:Anforderungen}
Anforderungen und Randbedingungen\index{Randbedingungen} \ldots

%% ==============================
\section{Existierende Lösungsansätze}
%% ==============================
\label{ch:Relativierung:sec:RelatedWork}

Hier kommt eine ausführliche Diskussion
von "`Related Work"'.

Bla fasel\ldots


%% ==============================
\section{Zusammenfassung}
%% ==============================
\label{ch:Relativierung:sec:zusammenfassung}

Am Ende sollten ggf. die wichtigsten Ergebnisse nochmal in \emph{einem}
kurzen Absatz zusammengefasst werden.

%%% Local Variables: 
%%% mode: latex
%%% TeX-master: "thesis"
%%% End: 
