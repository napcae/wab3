%% analyse.tex
%% $Id: analyse.tex 61 2012-05-03 13:58:03Z bless $

\chapter{Fehleranalyse}
\label{ch:Fehleranalyse}
%% ==============================
In diesem Kapitel sollten zunächst das zu lösende Problem
sowie die Anforderungen und die Randbedingungen 
einer Lösung\index{Lösung} beschrieben werden (also nochmal
eine präzisierte Aufgabenstellung\index{Aufgabenstellung}).

Dann folgt üblicherweise ein Überblick über bereits existierende
Lösungen bzw. Ansätze, die meistens andere Voraussetzungen bzw.
Randbedingungen annehmen.

Für eine aussagekräftige Analyse der Daten, sowie ein erfolgreiches Ende des Projekt ist es wichtig ein paar essentielle Rahmenbedingungen für die Probanden zu erstellen und eventuelle kritische Erfolgsfaktoren, die das Ergebnis dieses Projekts verfälschen können, festzuhalten.
So müssen alle Probanden täglich innerhalb der einmonatigen Testphase die unter \ref{ch:grundlagen:sec:qs-allgemein} genannten mobilen Applikationen nutzen, um die notwendigen Daten zu generieren. 
Sollte das nicht geschehen, wird das Ergebnis verfälscht und nicht eindeutig. Weiter werden die generierten Daten mindestens einmal in der Woche zusammengetragen und durch Zwischenanalysen erste Ergebnisse erbracht. 
Dies soll einem etwaigen Verlust wichtiger Daten vorbeugen und ein erstes Anzeichen auf das Endresultat darstellen. 
Sollte ein Proband die Nutzung der Applikationen an einem Tag vergessen haben, ist dies sofort festzuhalten und bei der Auswertung zu berücksichtigen. 
Ein kritischer Erfolgsfaktor ist ein während der Testphase erkrankter Proband: Sollte ein Proband während der Testphase krank werden, kann dies massiv die generierten Daten verfälschen, da eine Krankheit auf die beobachteten Faktoren (Schlaf, Bewegung und Ernährung) großen Einfluss hat. 
Die Daten, die während der Krankheitsphase gewonnen wurden müssen deswegen gesondert analysiert werden. 



%% ==============================
\section{Anforderungen}
%% ==============================
\label{ch:Analyse:sec:Anforderungen}
Anforderungen und Randbedingungen\index{Randbedingungen} \ldots

%% ==============================
\section{Existierende Lösungsansätze}
%% ==============================
\label{ch:Analyse:sec:RelatedWork}

Hier kommt eine ausführliche Diskussion
von "`Related Work"'.

Bla fasel\ldots

%% ==============================
\section{Weiterer Abschnitt}
%% ==============================
\label{ch:Analyse:sec:Abschnitt}

Bla fasel\ldots hat auch schon \cite{TB2000} gesagt und
\cite{TB98,JSAC96,qosr} sollte man mal gelesen haben.
Abbildung~\ref{fig:test} auf S.~\pageref{fig:test} sollte man
sich mal anschauen.

Blindtext Blindtext Blindtext Blindtext Blindtext Blindtext Blindtext
Blindtext Blindtext Blindtext Blindtext Blindtext Blindtext Blindtext
Blindtext Blindtext Blindtext Blindtext Blindtext Blindtext Blindtext
Blindtext Blindtext Blindtext Blindtext Blindtext Blindtext Blindtext
Blindtext Blindtext Blindtext Blindtext Blindtext Blindtext Blindtext
Blindtext Blindtext Blindtext Blindtext Blindtext Blindtext Blindtext
Blindtext Blindtext Blindtext Blindtext Blindtext Blindtext Blindtext
Blindtext Blindtext Blindtext Blindtext Blindtext Blindtext Blindtext
Blindtext Blindtext Blindtext Blindtext Blindtext Blindtext Blindtext

Blindtext Blindtext Blindtext Blindtext Blindtext Blindtext Blindtext
Blindtext Blindtext Blindtext Blindtext Blindtext Blindtext Blindtext
Blindtext Blindtext Blindtext Blindtext Blindtext Blindtext Blindtext
Blindtext Blindtext Blindtext Blindtext Blindtext Blindtext Blindtext
Blindtext Blindtext Blindtext Blindtext Blindtext Blindtext Blindtext
Blindtext Blindtext Blindtext Blindtext Blindtext Blindtext Blindtext
Blindtext Blindtext Blindtext Blindtext Blindtext Blindtext Blindtext
Blindtext Blindtext Blindtext Blindtext Blindtext Blindtext Blindtext
Blindtext Blindtext Blindtext Blindtext Blindtext Blindtext Blindtext
Blindtext Blindtext Blindtext Blindtext Blindtext Blindtext Blindtext
Blindtext Blindtext Blindtext Blindtext Blindtext Blindtext Blindtext
Blindtext Blindtext Blindtext Blindtext Blindtext Blindtext Blindtext
Blindtext Blindtext Blindtext Blindtext Blindtext Blindtext Blindtext

Blindtext Blindtext Blindtext Blindtext Blindtext Blindtext Blindtext
Blindtext Blindtext Blindtext Blindtext Blindtext Blindtext Blindtext
Blindtext Blindtext Blindtext Blindtext Blindtext Blindtext Blindtext
Blindtext Blindtext Blindtext Blindtext Blindtext Blindtext Blindtext
Blindtext Blindtext Blindtext Blindtext Blindtext Blindtext Blindtext
Blindtext Blindtext Blindtext Blindtext Blindtext Blindtext Blindtext
Blindtext Blindtext Blindtext Blindtext Blindtext Blindtext Blindtext
Blindtext Blindtext Blindtext Blindtext Blindtext Blindtext Blindtext
Blindtext Blindtext Blindtext Blindtext Blindtext Blindtext Blindtext
Blindtext Blindtext Blindtext Blindtext Blindtext Blindtext Blindtext

Blindtext Blindtext Blindtext Blindtext Blindtext Blindtext Blindtext
Blindtext Blindtext Blindtext Blindtext Blindtext Blindtext Blindtext
Blindtext Blindtext Blindtext Blindtext Blindtext Blindtext Blindtext
Blindtext Blindtext Blindtext Blindtext Blindtext Blindtext Blindtext
Blindtext Blindtext Blindtext Blindtext Blindtext Blindtext Blindtext
Blindtext Blindtext Blindtext Blindtext Blindtext Blindtext Blindtext
Blindtext Blindtext Blindtext Blindtext Blindtext Blindtext Blindtext
Blindtext Blindtext Blindtext\index{Blindtext} Blindtext Blindtext Blindtext Blindtext

\begin{figure}[!htbp]
  \centering
  \fbox{\parbox{0.8\textwidth}{
  Abbildungen sollten möglichst als EPS (Encapsulated Postscript) 
  bzw. PDF eingebunden werden.
  Zur Erzeugung sauberer EPS-Dateien empfiehlt sich das Tool \texttt{ps2eps}
  zur Nachbearbeitung von Postscript-Dateien. Mit \texttt{epstopdf} kann
  dann eine PDF-Datei zum Einbinden erzeugt werden.}}
  \caption{Testabbildung}
  \label{fig:test}
\end{figure}

Blindtext Blindtext Blindtext Blindtext Blindtext Blindtext Blindtext
Blindtext Blindtext Blindtext Blindtext Blindtext Blindtext Blindtext
Blindtext Blindtext Blindtext Blindtext Blindtext Blindtext Blindtext
Blindtext Blindtext Blindtext Blindtext Blindtext Blindtext Blindtext
Blindtext Blindtext Blindtext Blindtext Blindtext Blindtext Blindtext
Blindtext Blindtext Blindtext Blindtext Blindtext Blindtext Blindtext
Blindtext Blindtext Blindtext Blindtext Blindtext Blindtext Blindtext

Blindtext Blindtext Blindtext Blindtext Blindtext Blindtext Blindtext
Blindtext Blindtext Blindtext Blindtext Blindtext Blindtext Blindtext
Blindtext Blindtext Blindtext Blindtext Blindtext Blindtext Blindtext

Blindtext Blindtext Blindtext Blindtext Blindtext Blindtext Blindtext
Blindtext Blindtext Blindtext Blindtext Blindtext Blindtext Blindtext
Blindtext Blindtext Blindtext Blindtext Blindtext Blindtext Blindtext
Blindtext Blindtext Blindtext Blindtext Blindtext Blindtext Blindtext
Blindtext Blindtext Blindtext Blindtext Blindtext Blindtext Blindtext
Blindtext Blindtext Blindtext Blindtext Blindtext Blindtext Blindtext

Blindtext Blindtext Blindtext Blindtext Blindtext Blindtext Blindtext
Blindtext Blindtext Blindtext Blindtext Blindtext Blindtext Blindtext
Blindtext Blindtext Blindtext Blindtext Blindtext Blindtext Blindtext
Blindtext Blindtext Blindtext Blindtext Blindtext Blindtext Blindtext
Blindtext Blindtext Blindtext Blindtext Blindtext Blindtext Blindtext
Blindtext Blindtext Blindtext Blindtext Blindtext Blindtext Blindtext
Blindtext Blindtext Blindtext Blindtext Blindtext Blindtext Blindtext
Blindtext Blindtext Blindtext Blindtext Blindtext Blindtext Blindtext
Blindtext Blindtext Blindtext Blindtext Blindtext Blindtext Blindtext
Blindtext Blindtext Blindtext Blindtext Blindtext Blindtext Blindtext
Blindtext Blindtext Blindtext Blindtext Blindtext Blindtext Blindtext
Blindtext Blindtext Blindtext Blindtext Blindtext Blindtext Blindtext
Blindtext Blindtext Blindtext Blindtext Blindtext Blindtext Blindtext
Blindtext Blindtext Blindtext Blindtext Blindtext Blindtext Blindtext

Blindtext Blindtext Blindtext Blindtext Blindtext Blindtext Blindtext
Blindtext Blindtext Blindtext Blindtext Blindtext Blindtext Blindtext
Blindtext Blindtext Blindtext Blindtext Blindtext Blindtext Blindtext
Blindtext Blindtext Blindtext Blindtext Blindtext Blindtext Blindtext
Blindtext Blindtext Blindtext Blindtext Blindtext Blindtext Blindtext
Blindtext Blindtext Blindtext Blindtext Blindtext Blindtext Blindtext
Blindtext Blindtext Blindtext Blindtext Blindtext Blindtext Blindtext
Blindtext Blindtext Blindtext Blindtext Blindtext Blindtext Blindtext
Blindtext Blindtext Blindtext Blindtext Blindtext Blindtext Blindtext
Blindtext Blindtext Blindtext Blindtext Blindtext Blindtext Blindtext
Blindtext Blindtext Blindtext Blindtext Blindtext Blindtext Blindtext
%% ==============================
\section{Zusammenfassung}
%% ==============================
\label{ch:Analyse:sec:zusammenfassung}

Am Ende sollten ggf. die wichtigsten Ergebnisse nochmal in \emph{einem}
kurzen Absatz zusammengefasst werden.

%%% Local Variables: 
%%% mode: latex
%%% TeX-master: "thesis"
%%% End: 
