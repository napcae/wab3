% !TEX encoding = UTF-8 Unicode
\documentclass{wissdoc}
% Autor: Chi Trung Nguyen 2013-2014, chitrungnguyen <at> me.com
% Vorlagenautor: Roland Bless 1996-2009, bless <at> kit.edu
% ----------------------------------------------------------------
% Diplomarbeit - Hauptdokument
% ----------------------------------------------------------------
%%
%% $Id: thesis.tex 65 2012-05-10 10:32:11Z bless $
%%
% wissdoc Optionen: draft, relaxed, pdf --> siehe wissdoc.cls
% ------------------------------------------------------------------
% Weitere packages: (Dokumentation dazu durch "latex <package>.dtx")
\usepackage[numbers,sort&compress]{natbib}
\usepackage{float}
\restylefloat{table} %!damit tabellen dort auftauchen, wo sie beschrieben werden([H] benutzen!)
\usepackage[official]{eurosym} % fÃŒr Eurosymbol
\usepackage{listings} %fÃŒr Quellcode
\lstset{
  literate={Ö}{{\"O}}1
{Ä}{{\"A}}1
{Ü}{{\"U}}1
{ß}{{\ss}}2
{Ì}{{\"u}}1
{À}{{\"a}}1
{ö}{{\"o}}1
}
% \usepackage{varioref}
% \usepackage{verbatim}
% \usepackage{float}    %z.B. \floatstyle{ruled}\restylefloat{figure}
% \usepackage{subfigure}
% \usepackage{fancybox} % fÃŒr schattierte,ovale Boxen etc.
% \usepackage{tabularx} % automatische Spaltenbreite
% \usepackage{supertab} % mehrseitige Tabellen
% \usepackage[svnon,svnfoot]{svnver} % SVN Versionsinformation 
%% ---------------- end of usepackages -------------

%\svnversion{$Id: thesis.tex 65 2012-05-10 10:32:11Z bless $} % In case that you want to include version information in the footer

%% Informationen fÃŒr die PDF-Datei
\hypersetup{
 pdfauthor={Andreas Hornig, Florian Weber, Chi Trung Nguyen},
 pdftitle={Not set}
 pdfsubject={Not set},
 pdfkeywords={Not set}
}

% Macros, nicht unbedingt notwendig
\input{macros}

% Print URLs not in Typewriter Font
\def\UrlFont{\rm}

\newcommand{\blankpage}{% Leerseite ohne Seitennummer, nÀchste Seite rechts
 \clearpage{\pagestyle{empty}\cleardoublepage}
}

%% Einstellungen fÃŒr das gesamte Dokument

% Trennhilfen
% Wichtig! 
% Im ngerman-paket sind zusÀtzlich folgende Trennhinweise enthalten:
% "- = zusÀtzliche Trennstelle
% "| = Vermeidung von Ligaturen und mögliche Trennung (bsp: Schaf"|fell)
% "~ = Bindestrich an dem keine Trennung erlaubt ist (bsp: bergauf und "~ab)
% "= = Bindestrich bei dem Worte vor und dahinter getrennt werden dÃŒrfen
% "" = Trennstelle ohne Erzeugung eines Trennstrichs (bsp: und/""oder)

% Trennhinweise fuer Woerter hier beschreiben
\hyphenation{
% Pro-to-koll-in-stan-zen
% Ma-na-ge-ment  Netz-werk-ele-men-ten
% Netz-werk Netz-werk-re-ser-vie-rung
% Netz-werk-adap-ter Fein-ju-stier-ung
% Da-ten-strom-spe-zi-fi-ka-tion Pa-ket-rumpf
% Kon-troll-in-stanz
}

% Index-Datei öffnen
\ifnotdraft{\makeindex}
%%%%%%%%%%%%%% includeonly %%%%%%%%%%%%%%%%%%%
% Es werden nur die Teile eingebunden, die hier 
% aufgefuehrt sind!
\includeonly{%
titelseite,%
erklaerung,% Ist in KA Pflicht fÃŒr Diplomarbeiten
einleitung,% Motivation, Zielsetzung, Gliederung
grundlagen,% Grundlagen 
apps,
fehleranalyse, %Relativierung, mögliche Fehlerquellen (technische, persönliche, falsche Wahrnehmung)
eval,
analyse,   % Problembeschreibung (Detail) und Related Work
zusammenf  % Zusammenfassung der Ergebnisse und Ausblick
}
%%%%%%%%%%%%%%%%%%%%%%%%%%%%%%%%%%%%%%%%%%%%%%
\begin{document}

\frontmatter
\pagenumbering{roman}
\ifnotdraft{
 %% Titelseite
%% Vorlage $Id: titelseite.tex 61 2012-05-03 13:58:03Z bless $

\def\usesf{}
\let\usesf\sffamily % diese Zeile auskommentieren für normalen TeX Font

\newsavebox{\Erstgutachter}
\savebox{\Erstgutachter}{\usesf Prof.~Dr.-Ing. Oliver Jokisch}
\newsavebox{\Zweitgutachter}
\savebox{\Zweitgutachter}{\usesf Prof.~Dr.-Ing. Undine Pielot}

\begin{titlepage}
\setlength{\unitlength}{1pt}
\begin{picture}(0,0)(85,770)
\includegraphics[width=\paperwidth]{logos/HFTL_Deckblatt}
\end{picture}

\thispagestyle{empty}

%\begin{titlepage}
%%\let\footnotesize\small \let\footnoterule\relax
\begin{center}
\hbox{}
\vfill
{\usesf
{\huge\bfseries Lässt sich das Wohlbefinden durch Quantified Self\\
                verbessern?\par}
\vskip 1.8cm
Projektbericht WAB 3\\
von\\[2mm]
\vskip 1cm

{\large\bfseries Chi Trung Nguyen\\
Florian Weber\\
Andreas Hornig}
\vskip 1.2cm
an der Hochschule für Telekommunikation\\
\vskip 3cm
\begin{tabular}{p{5.5cm}l}
Erstgutachter: & \usebox{\Erstgutachter} \\
Zweitgutachter: & \usebox{\Zweitgutachter} \\
\end{tabular}
\vskip 3cm
Bearbeitungszeit:\qquad 15. April 2014 -- 29. Juni~2014
}
\end{center}
\vfill
\end{titlepage}
%% Titelseite Ende


%%% Local Variables: 
%%% mode: latex
%%% TeX-master: "thesis"
%%% End: 

 \blankpage % Leerseite auf TitelrÃŒckseite
 %
 % Die folgende ErklÀrung ist fÌr Diplomarbeiten Pflicht
 % (siehe PrÃŒfungsordnung), fÃŒr Studienarbeiten nicht notwendig
 \thispagestyle{empty}
\vspace*{42\baselineskip}
\hbox to \textwidth{\hrulefill}
\par
Wir erklären hiermit, dass wir die vorliegende Arbeit selbständig verfasst und
keine anderen als die angegebenen Quellen und Hilfsmittel verwendet haben.

München, den 26. Juni 2014

%%%%%%%%%%%%%%%%%%%%%%%%%%%%%%%%%%%%%%%%%%%%%%%%%%%%%%%%%%%%%%%%%%%%%%%%
%% Hinweis:
%%
%% Diese Erklärung wird von der Prüfungsordnung für Diplomarbeiten 
%% verlangt und ist zu unterschreiben. Für Studienarbeiten ist diese
%% Erklärung nicht zwingend notwendig, schadet aber auch nicht.
%%%%%%%%%%%%%%%%%%%%%%%%%%%%%%%%%%%%%%%%%%%%%%%%%%%%%%%%%%%%%%%%%%%%%%%%
\clearpage







 \blankpage % Leerseite auf ErklÀrungsrÌckseite
}
%
%% *************** Hier geht's ab ****************
%% ++++++++++++++++++++++++++++++++++++++++++
%% Verzeichnisse
%% ++++++++++++++++++++++++++++++++++++++++++
\ifnotdraft{
{\parskip 0pt\tableofcontents} % toc bitte einzeilig
\blankpage
%\listoffigures
%\blankpage
%\listoftables
%\blankpage
}


%% ++++++++++++++++++++++++++++++++++++++++++
%% Hauptteil
%% ++++++++++++++++++++++++++++++++++++++++++
\graphicspath{{Bilder/}}

\mainmatter
\pagenumbering{arabic}
%% Einleitung.tex
%% $Id: einleitung.tex 61 2012-05-03 13:58:03Z bless $
%%

\chapter{Einleitung}
\label{ch:Einleitung}
%% ==============================

%% ==============================
\section{Problemstellung und Zielsetzung}
%% ==============================
\label{ch:Einleitung:sec:problemstellung-und-zielsetzung}

Seit Gründung der Initiative \href{http://quantifiedself.com/}{\textbf{Quantified Self}} im \href{http://quantifiedself.com/2011/03/what-is-the-quantified-self/}{\textbf{Jahr 2007}} steigen die Möglichkeiten von Jahr zu Jahr, Umwelt und Personenbezogne Daten zu erfassen. 

Dies wird durch unterschiedliche Hard und Softwarelösungen ermöglicht. 

Dabei werden Erkenntnisse über Gesundheit, Fitness und persönliche Stimmung gesammelt.

Diese können auch zu externen Umweltfaktoren in Relation gebracht werden.

Als Leitfrage des Projekts wurde die Frage, ob durch Qantified Self das Leben verbessert werden kann, festgesetzt. Das Erreichen des Ziels, die Beantwortung der Leitfrage durch das Analysieren und Auswerten von aus Selbstversuchen gewonnener Daten.

Zur Zielerreichung wird zu Beginn der Datengenerierungs- bzw. Testphase, die 30 Tage beträgt, der augenblickliche Zustand der Probanden aufgezeichnet und gesichert - also der derzeitige Schlafrhythmus, derzeitige Essgewohnheit und Bewegungsaktivität. 

Dieser wird als 100\% Marke angesetzt und dient der späteren Auswertung der gewonnen Daten als Maßstab. Die Daten werden aus Bewegungsaktivität, Schlafrhythmus und Stimmungslage gewonnen. 

Sollte der analysierte Wert nach der Testphase über dieser Marke liegen, liegt eindeutig eine Verbesserung vor. 

Ist der Wert darunter, so stellt dieser eine Verschlechterung dar. 

Zur besseren Klassifizierung der Daten wird von einer Verbesserung erst ab dem Wert von mindestens 120\% gesprochen, sowie von einer Verschlechterung bei einem Wert von 80\%. Sollte der Endwert eines Probanden zwischen 80\% und 120\% liegen wird von einem Gleichbleiben des Befindens gesprochen.

In der heutigen Zeit, in der die Lebenssituation, v.a. in der arbeitenden Bevölkerung, an Qualität abnimmt – sei es durch Stress im Arbeitsalltag oder der gewaltigen Informationsflut, die uns unterbewusst immer und überall beeinträchtigt – ist es wichtig, neue Möglichkeiten auszuloten, um die Lebensqualität z.B. durch die Selbstanalyse diverser Faktoren wieder zu verbessern. 

In diesem Projekt werden Faktoren wie Schlaf, Ernährung und Bewegungsaktivität sein, die mit Hilfe von Quantified Self Appliaktionen für das Smartphone aufgezeichnet und später analysiert werden. 

Dadurch soll herausgefunden werden, ob eine Verbesserung durch die Nutzung von QS-Applikationen möglich ist.

Die stetig steigende Anzahl von Burnout-Patienten und die Selbsteinschätzung vieler Menschen in Deutschland, die entgegen dem eigentlichen Trend, eine sinkende Lebensqualität bemängeln, versuchen wir mit unserem Projekt eine Perspektive zu geben, wie man eventuell die Situation durch den Einsatz mobiler QS-Applikation für diverse Faktoren verbessern kann. 

Dieses Projekt soll eventuelle neue Möglichkeiten zur Verbesserung des Lebens durch das Nutzen von QS aufzeigen und helfen den Burnout zu verhindern bzw. Stress abzubauen und so das Gesundheitssystem teilweise entlasten, sowie das Lebensgefühl verbessern. 

%% ==============================
\section{Gliederung der Arbeit}
%% ==============================
\label{ch:Einleitung:sec:gliederung-der-arbeit}

Die Arbeit ist in sieben Teile gegliedert:

\begin{enumerate}
\def\labelenumi{\arabic{enumi}.}
\itemsep1pt\parskip0pt\parsep0pt
\item
  Einleitung (Motivation, Trend)
\item
  Informationen zu Quantified Self (Studien, Trend, Medizinische
  Untersuchung)
\item
  Softwarebeschreibung (Erläuterung, Einführung)

  \begin{enumerate}
  \def\labelenumii{\alph{enumii}.}
  \itemsep1pt\parskip0pt\parsep0pt
  \item
    Moves (Bewegungsaktivität)\\
  \item
    Hueman (pers. Wohlbefinden)\\
  \item
    SleepCycle (Schlafzyklen-Analyse)
  \end{enumerate}
\item
  Relativierung: Mögliche Fehlerquellen (Technische, Persönliche,
  Falsche Wahrnehmung der eigenen Verfassung)
\item
  Auswertung der generierten App-Daten
\item
  Analyse der ausgewerteten Daten
\item
  Fazit (Beantwortung der Leitfrage)
\end{enumerate}

Die Einleitung soll einen Einblick in die Problemstellung und
Zielsetzung der Arbeit, Motivation und Trend, sowie den Aufbau der
Arbeit beschreiben.

Informationen zu Quantified Self gibt Aufschluss über aktuelle Studien
zu Quantified Self sowie den Trend und Medizinische Untersuchungen.

Innerhalb die Softwarebeschreibung wird detailierter auf die Auswahl der
Apps eingegangen. Zusätzlich sind deren Funktionsweise und Features hier
beschrieben.

Die Relativierung beschreibt mögliche Technische und Persönliche
Fehlerquellen bei der Andwendung, sowie die Problematik bei Falscher
Warnehmung der eigenen Verfassung.

Auswertung der generierten App-Daten

Analyse der ausgewerteten Daten

Das Fazit beantwortet die Leitfrage des Projektes und soll Aufschluss
über mögliche Verbesserungsideen geben.

%% ==============================
\section{Auswahl der Trackingmethoden}
%% ==============================
\label{ch:Einleitung:sec:auswahl-der-trackingmethoden}

Aufgrund der gegeben Mittel und dem Ziel die Fragestellung realitätsnah zu beantworten, beschränken wir unsere Trackingmethoden auf reine Softwarelösungen. Diese können mit etlichen Handelsüblichen Smartphones benutzt werden und liefern für weniger als 2\euro{} gute Ergebnisse. Die Arbeit richtet sich orientiert sich an alltagsüblichen Situtation. Daher benutzt das Projektteam einen Schrittzähler ("Moves"), Schlafrzykluserfassung ("Sleep Cycle") und einen Stimmungsbarometer ("Human").
Die Software wird im folgenden näher erläutert.

Die Zielgruppe, für die dieses Projekt ins Leben gerufen wurde, sind vor allem Smartphone-Nutzer, deren derzeitiges Leben, sei es durch Stress im Arbeitsalltag oder Burnout-ähnlichen Symptomen, verbesserungswürdig ist bzw. die die derzeitige Lebenssituation zu verbessern suchen.  

%%% Local Variables: 
%%% mode: latex
%%% TeX-master: "thesis"
%%% End: 
  % Einleitung
%% grundlagen.tex
%% $Id: grundlagen.tex 61 2012-05-03 13:58:03Z bless $
%%

\chapter{Grundlagen}
\label{ch:Grundlagen}
%% ==============================
Die Grundlagen müssen soweit beschrieben werden, dass ein Leser das Problem und die Problemlösung versteht.
Um nicht zuviel zu beschreiben, kann man das auch erst gegen Ende der Arbeit schreiben.


%% ==============================
\section{Quantified Self}
%% ==============================
\label{ch:Grundlagen:sec:QuantifiedSelf}

\subsection{Allgemein}
%% ==============================
\label{ch:Grundlagen:sec:QuantifiedSelf:subsec:Allgemein}

Quantified Self ist ein Ausdruck dafür, Technologien oder Anwendungen zu nutzen, um das tägliche Leben durch ständige Datenerfassung zu visualisieren.
Hauptsächliche Bereiche, die durch Quantified Self erfasst werden sollen sind z.B. Essgewohnheit (welche Lebensmittel verzehrt wurden), persönliches Wohlbefinden (Stimmung, Erregung, Sauerstoffgehalt im Blut) und die Leistung (mentale und physische). 
Umgangssprachlich wird eine solche Selbstüberwachung und -erkennung, die tragbare Sensoren (EEG, ECG-, Video-, etc.) und mobile Plattformen (tragbare Fittness-Gadgets, Smartphones/Tablets, etc.) verbindet, auch „self-tracking“ genannt. 
Durch den heutigen Stand der Technik ist es so jedem möglich, bisher unbekannte eigene biometrische Daten kostengünstig und bequem zu ermitteln.

\subsection{Einsatzgebiete}
%% ==============================
\label{ch:Grundlagen:sec:QuantifiedSelf:subsec:Einsatzgebiete}

Der Hauptanwendungsbereich von Quantified Self ist die Verbesserung der eigenen Gesundheit und des persönlichen Wohlbefindens. 
Für diesen Bereich gibt es viele Geräte und Applikationen, die die körperliche Aktivität, die Kalorienzufuhr, die Schlafqualität, die Körperhaltung und andere Faktoren des persönlichen Wohlbefindens analysieren und helfen, die gewonnenen Daten visuell verständlich darzulegen. 
Diese Gesundheitsüberwachung soll das persönliche Wohlbefinden des Nutzers aufrecht erhalten und so potentielle Krankheiten verhindern.
Die daraus resultierenden sinkenden Gesundheitskosten sind vor allem für Nutzer in Ländern ohne öffentliches Gesundheitssystem eine Option für Nutzung.
\\
\\
Ein weiteres Anwendungsgebiet findet sich in der Bildung. 
So nutzen viele Schulen, vor allem in den USA, QS-Applikationen, um den Schülern „schwierige“ Fächer praxisnah beizubringen. 
So werden dort die Aktivitäten der Schüler aufgenommen und ausgewertet.
Aus den gewonnenen Daten werden themenrelevante Gebiete aus der Mathematik und den Naturwissenschaften den Schülern direkt auf das Smartphone geliefert. 

\ldots 


\subsection{Ausblick}
%% ==============================
\label{ch:Grundlagen:sec:QuantifiedSelf:subsec:Ausblick}

Quantified Self ist in Deutschland noch in den Kinderschuhen.
Die Bewegung des Self-Tracking wächst langsam aber stetig weiter an und entwickelt sich weiter.
So ist es durchaus denkbar, dass aus Quantified Self  ein „Quantified Us“ entstehen wird.
Da das eigene Tracking eine starke Disziplin des Nutzers verlang und keinen Vergleich mit anderen Personen zulässt ist, wäre die Entwicklung zu einem „Us“ nur zu gut nachvollziehbar.
\\
Stellt man sich beispielsweise eine Person mit Epilepsie vor, die versucht die getrackte Steigerung der Anfälle verstehen will. 
Könnte er die ausgewerteten Daten besser verstehen, wenn sie im Vergleich zu den Daten anderen Personen mit der selben Krankheit stünden ? 
Eine solche Möglichkeit könnte den Nutzern sehrwohl weiterbringen, da dies erstens andere Geräte und Anwendungen überflüssig macht. 
Und zweitens dem Nutzer verdeutlicht, ob seine Unfallhäufigkeit im Durchschnitt zu anderen liegt.
Das gepaart mit der Kontaktaufnahme zu anderen Betroffenen könnte einen zusätzlichen Schub an Motivation mit sich bringen. 
So wird die letztendliche Form des Quantified Self das Quantified Us werden, ein soziales self-tracking Netzwerk, das den Austausch zwischen Kranken maßgeblich erleichtern  und ein Verbundenheitsgefühl unter diesen etablieren wird.


% %% ==============================
% \section{Abschnitt 2}
% %% ==============================
% \label{ch:Grundlagen:sec:Abschnitt2}

% Bla fasel\ldots

%% ==============================
\section{Verwandte Arbeiten}
%% ==============================
\label{ch:Grundlagen:sec:RelatedWork}
Hier kommt "`Related Work"' rein.
Eine Literaturrecherche sollte so vollständig wie möglich sein, relevante Ansätze müssen beschrieben werden und es sollte deutlich gemacht werden, wo diese Ansätze Defizite aufweisen oder nicht anwendbar sind, z.\,B. weil sie von anderen Umgebungen oder Voraussetzungen ausgehen.


Bla fasel\ldots

%%% Local Variables: 
%%% mode: latex
%%% TeX-master: "thesis"
%%% End: 
  % Grundlagen
%% Einleitung.tex
%% $Id: einleitung.tex 61 2012-05-03 13:58:03Z bless $
%%

\chapter{Auswahl der Applikationen}
\label{ch:Apps}
%% ==============================
Unsere Auswahl der Apps
Erläuterung, Einführung
Bla fasel\ldots



%% ==============================
\section{Moves}
%% ==============================
\label{ch:Apps:sec:Moves}

Die Aktivitäts- und Location-Tracking App **Moves** soll ein automatisches Tagebuch sein, dass dem Nutzer sagt und zeigt, wo, wann und vor allem wie lang er in Bewegung war.

Dafür analysiert die App automatisch alle Lauf-, Radfahr- und Rennaktivität, nimmt Sie auf und speichert sie. 

Die gespeicherten Daten, wie z.B. die Distanz, die Dauer und die Anzahl der Schritte  sowie der Kalorienverbrauch jeder Aktivität, werden grafisch für den Benutzer übersichtlich dargestellt. 

Damit die App die Daten generieren kann ist sie „Always-On“ – also ständig mit dem Internet verbunden und es ist nicht nötig, die App zu Starten oder zu beenden.

Die einzelnen Funktionen der App sind: 

\subsection{Der Schrittzähler}

Der Schrittzähler in **Moves** ist die Hauptfunktion der App und zeigt dem Nutzer, wie viele Schritte er täglich gegangen ist. 

**Moves** nimmt neben dem normalen Laufen auch Fahrradfahren, Joggen oder die Fahrten mit anderen Verkehrsmitteln zu Kenntnis. 

Laut der App sind täglich 10.000 Schritte das Minimum, das ein gesundheitsbewusster Mensch erreichen sollte.

\begin{figure}[H]
\centering
	\includegraphics[width=0.6\textwidth]{images/moves_app_screenshot} 
	\caption[Screenshot des Moves Bildschirms]{Screenshot des Moves Bildschirms}
	\label{fig:moves_screenshot}
\end{figure}

\subsection{Der Kalorienzähler}

Mit dem Kalorienzähler der in der App zusätzlich beinhaltet ist, werden die Kalorien ermittelt, die in der jeweiligen Aktivität verbrannt wurden. 
Dazu kann dem Nutzer die tägliche optimale Menge der zu verbrennenden Kalorien angezeigt werden.

\subsection{Third-Party-Apps-Integration}

Da die App nicht nur die eigenen Dienste unterstützt, sondern auch Apps und Services von Drittanbietern. 
So existieren seit Kurzem die Möglichkeit, bis zu 15 andrere Apps in Moves zu integrieren. 
Der Start der API, die es Entwicklern und Anbietern von Third-Party-Services bzw. –Apps erlaubt, **Moves** einzubinden, war der erste Schritt, um die gesammelten Daten umfassend analysieren und auswerten zu können.

%% ==============================
\section{Hueman}
%% ==============================
\label{ch:Apps:sec:Hueman}

Beschreibung von Hueman

Bla fasel\ldots

%% ==============================
\section{Sleep Cycle}
%% ==============================
\label{ch:Apps:sec:SleepCycle}

Wir bewegen uns in den verschieden Phasen des Schlafes unterschiedlich stark. Die Software „Sleep Cycle“ benutzt daher das eingebaute „Accelerometer“ (dt. Beschleunigungssensor) des Aufzeichungsgerätes (Smartphone), um die diese Bewegungen zu erfassen. Die Bewegungen sind ausschlaggebend für die Bestimmung der Zustandsphasen des Schlafes. Der Alarm des integrierten Wecker ertönt, wenn sich der benutzer in der Leichtesten Phase des Schlafes befindent. Man fühlt sich ausgeschlafener und erholter, wenn man in den Phasen des leichten Schlafes geweckt wird (*Belegen*). 

Zudem bietet „Sleep Cylce“ ein Tracking und Analyse Funktion. Diese zeigen dem Nutzer Informationen über deren Schlaf sowie Mögliche Ursachen für Störung. Dafür analysiert die Software die Bewegungen während des Schlafes und erfasst Informationen über den Vergangen Tag des Benutzers. Alle erfassten Daten werden in unterschiedlichen Grafiken und Plots dargestellt.

\section{Funktionsweise und Anwendung}

Vor Beginn des Schlafes, wird die gewünschte Alarmzeit konfiguriert und das Aufzeichnungsgerät, gemäß Softwareentwicker, auf dem Bett neben dem Kopfkissen platziert. 

Während des Schlafes durchläuft der Organismus mehrmals verschiedene Stadien. Diese lauten (*Belegen*):

\begin{itemize}
	\item Stadium 1
	\item Stadium 2
	\item Stadium 3
	\item Stadium 4
	\item R.E.M. Phase
\end{itemize}


%%% Local Variables: 
%%% mode: latex
%%% TeX-master: "thesis"
%%% End: 

\include{fehleranalyse} %Fehleranalyse
\include{eval}        % Evaluierung
%% analyse.tex
%% $Id: analyse.tex 61 2012-05-03 13:58:03Z bless $

\chapter{Analyse und Evaluierung}
\label{ch:AnalyseUndEvaluierung}
%% ==============================

Die Beantwortung der Leitfrage des Projekts, ob durch Quantified Self das Leben verbessert werden kann, hängt maßgeblich von einer genauen Analyse der Daten ab. 
Die zuvor in der Datengenerierungsphase erhaltenen Daten müssen unter Berücksichtigung etwaiger Fremdeinflüsse bzw. Verfälschungen analysiert und ausgewertet werden.
Um dies zu gewährleisten sind wichtige Anforderungen an die Analyse gestellt, auf die im Folgenden näher eingegangen wird.

%%In diesem Kapitel sollten zunächst das zu lösende Problem sowie die Anforderungen und die Randbedingungen einer Lösung\index{Lösung} beschrieben werden (also nochmal eine präzisierte Aufgabenstellung\index{Aufgabenstellung}). 
%%Dann folgt üblicherweise ein Überblick über bereits existierende Lösungen bzw. Ansätze, die meistens andere Voraussetzungen bzw. Randbedingungen annehmen.

Bla fasel\ldots

%% ==============================
\section{Anforderungen}
%% ==============================
\label{ch:AnalyseUndEvaluierung:sec:Anforderungen}
Anforderungen und Randbedingungen\index{Randbedingungen} \ldots

%% ==============================
\section{Existierende Lösungsansätze}
%% ==============================
\label{ch:AnalyseUndEvaluierung:sec:RelatedWork}

Hier kommt eine ausführliche Diskussion
von "`Related Work"'.

Bla fasel\ldots

%% ==============================
\section{Zusammenfassung}
%% ==============================
\label{ch:Analyse:sec:zusammenfassung}

Am Ende sollten ggf. die wichtigsten Ergebnisse nochmal in \emph{einem}
kurzen Absatz zusammengefasst werden.

%%% Local Variables: 
%%% mode: latex
%%% TeX-master: "thesis"
%%% End: 
     % Analyse
%% zusammenf.tex
%% $Id: zusammenf.tex 61 2012-05-03 13:58:03Z bless $
%%

\chapter{Zusammenfassung und Ausblick}
\label{ch:Zusammenfassung}
%% ==============================

Fazit:
Ob sich das Wohlbefinden durch Quantified Self verbessern lässt, kann man nicht durch Zahlen belegen. Es hängt von einem Selbst ab, inwiefern man sich damit besser fühlt.
*anregung motivation, interne konkurrenz(mehr laufen im projekt), ob es wirklich dazu beiträgt dass man gesünder ist ist eine andere frage, aber gefühlt fühlt man sich besser.
ziel von QS ist es nicht das leben zu verbessern, sonder dass man sich selbst besser fühlt. in sofern lässt sich die frage ob sich das wohlbefinden durch QS verbesern lässt mich ja beantworten.

%%% Local Variables: 
%%% mode: latex
%%% TeX-master: "thesis"
%%% End: 
   % Zusammenfassung und Ausblick

%% ++++++++++++++++++++++++++++++++++++++++++
%% Anhang
%% ++++++++++++++++++++++++++++++++++++++++++
\phantomsection
\appendix
%\include{
%\input{anhang_a}
%\input{anhang_b}

%% ++++++++++++++++++++++++++++++++++++++++++
%% Literatur
%% ++++++++++++++++++++++++++++++++++++++++++
%  mit dem Befehl \nocite werden auch nicht 
%  zitierte Referenzen abgedruckt
\cleardoublepage
\phantomsection
\addcontentsline{toc}{chapter}{\bibname}
% $ bibtex thesis
%%
%\nocite{*} % nur angeben, wenn auch nicht im Text zitierte Quellen 
           % erscheinen sollen
\bibliographystyle{ieeetr} % mit abgekÃŒrzten Vornamen der Autoren
%\bibliographystyle{gerplain} % abbrvnat unsrtnat
%\bibliographystyle{plainurl} % abbrvnat unsrtnat
% spezielle Zitierstile: Labels mit vier Buchstaben und Jahreszahl
%\bibliographystyle{itmalpha}  % ausgeschriebene Vornamen der Autoren
\bibliography{thesis}

\listoffigures
\listoftables
%% ++++++++++++++++++++++++++++++++++++++++++
%% Index
%% ++++++++++++++++++++++++++++++++++++++++++
\ifnotdraft{
\cleardoublepage
\phantomsection
\printindex            % Index, Stichwortverzeichnis
}
\end{document}
%% end of file
